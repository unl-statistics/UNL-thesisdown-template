% From https://github.com/UWIT-IAM/UWThesis
\documentclass[print]{nuthesis}
\usepackage{amssymb, amsthm, amsmath, amsfonts}
\usepackage{wasysym}
\usepackage{mathrsfs}
% \usepackage{hyperref}
\usepackage{graphicx}
\usepackage{lineno}
\usepackage[colorinlistoftodos]{todonotes}
\usepackage{listings}
%\usepackage{breqn}
\usepackage{cancel, enumerate}
\usepackage{rotating, environ}
\usepackage{caption}
\usepackage{subcaption}
\usepackage[inline]{enumitem}
\usepackage{dirtree}

\newtheorem{thm}{Theorem}
\newtheorem{defn}{Definition}
\newtheorem{prop}{Proposition}
\newtheorem{lemma}{Lemma}
\newtheorem{cor}{Corollary}

% Syntax highlighting #22
  \usepackage{color}
  \usepackage{fancyvrb}
  \newcommand{\VerbBar}{|}
  \newcommand{\VERB}{\Verb[commandchars=\\\{\}]}
  \DefineVerbatimEnvironment{Highlighting}{Verbatim}{commandchars=\\\{\}}
  % Add ',fontsize=\small' for more characters per line
  \usepackage{framed}
  \definecolor{shadecolor}{RGB}{248,248,248}
  \newenvironment{Shaded}{\begin{snugshade}}{\end{snugshade}}
  \newcommand{\AlertTok}[1]{\textcolor[rgb]{0.94,0.16,0.16}{#1}}
  \newcommand{\AnnotationTok}[1]{\textcolor[rgb]{0.56,0.35,0.01}{\textbf{\textit{#1}}}}
  \newcommand{\AttributeTok}[1]{\textcolor[rgb]{0.77,0.63,0.00}{#1}}
  \newcommand{\BaseNTok}[1]{\textcolor[rgb]{0.00,0.00,0.81}{#1}}
  \newcommand{\BuiltInTok}[1]{#1}
  \newcommand{\CharTok}[1]{\textcolor[rgb]{0.31,0.60,0.02}{#1}}
  \newcommand{\CommentTok}[1]{\textcolor[rgb]{0.56,0.35,0.01}{\textit{#1}}}
  \newcommand{\CommentVarTok}[1]{\textcolor[rgb]{0.56,0.35,0.01}{\textbf{\textit{#1}}}}
  \newcommand{\ConstantTok}[1]{\textcolor[rgb]{0.00,0.00,0.00}{#1}}
  \newcommand{\ControlFlowTok}[1]{\textcolor[rgb]{0.13,0.29,0.53}{\textbf{#1}}}
  \newcommand{\DataTypeTok}[1]{\textcolor[rgb]{0.13,0.29,0.53}{#1}}
  \newcommand{\DecValTok}[1]{\textcolor[rgb]{0.00,0.00,0.81}{#1}}
  \newcommand{\DocumentationTok}[1]{\textcolor[rgb]{0.56,0.35,0.01}{\textbf{\textit{#1}}}}
  \newcommand{\ErrorTok}[1]{\textcolor[rgb]{0.64,0.00,0.00}{\textbf{#1}}}
  \newcommand{\ExtensionTok}[1]{#1}
  \newcommand{\FloatTok}[1]{\textcolor[rgb]{0.00,0.00,0.81}{#1}}
  \newcommand{\FunctionTok}[1]{\textcolor[rgb]{0.00,0.00,0.00}{#1}}
  \newcommand{\ImportTok}[1]{#1}
  \newcommand{\InformationTok}[1]{\textcolor[rgb]{0.56,0.35,0.01}{\textbf{\textit{#1}}}}
  \newcommand{\KeywordTok}[1]{\textcolor[rgb]{0.13,0.29,0.53}{\textbf{#1}}}
  \newcommand{\NormalTok}[1]{#1}
  \newcommand{\OperatorTok}[1]{\textcolor[rgb]{0.81,0.36,0.00}{\textbf{#1}}}
  \newcommand{\OtherTok}[1]{\textcolor[rgb]{0.56,0.35,0.01}{#1}}
  \newcommand{\PreprocessorTok}[1]{\textcolor[rgb]{0.56,0.35,0.01}{\textit{#1}}}
  \newcommand{\RegionMarkerTok}[1]{#1}
  \newcommand{\SpecialCharTok}[1]{\textcolor[rgb]{0.00,0.00,0.00}{#1}}
  \newcommand{\SpecialStringTok}[1]{\textcolor[rgb]{0.31,0.60,0.02}{#1}}
  \newcommand{\StringTok}[1]{\textcolor[rgb]{0.31,0.60,0.02}{#1}}
  \newcommand{\VariableTok}[1]{\textcolor[rgb]{0.00,0.00,0.00}{#1}}
  \newcommand{\VerbatimStringTok}[1]{\textcolor[rgb]{0.31,0.60,0.02}{#1}}
  \newcommand{\WarningTok}[1]{\textcolor[rgb]{0.56,0.35,0.01}{\textbf{\textit{#1}}}}

%% https://github.com/rstudio/rmarkdown/issues/1649
\newlength{\cslhangindent}
\setlength{\cslhangindent}{1.5em}
\newenvironment{CSLReferences}[2]%
{\setlength{\parindent}{0pt}%
\everypar{\setlength{\hangindent}{\cslhangindent}}\ignorespaces}%
{\par}

% fix for pandoc 1.14
\providecommand{\tightlist}{%
  \setlength{\itemsep}{0pt}\setlength{\parskip}{0pt}}

%% something about tables, from https://github.com/ismayc/thesisdown/issues/122
\usepackage{calc}

%% for copyright symbol
\usepackage{textcomp}

%% to allow to rotate pages to landscape
\usepackage{lscape}
%% to adjust table column width
\usepackage{tabularx}

% suppress bottom page numbers on first page of each chapter
% because they overlap with text
\usepackage{etoolbox}
\patchcmd{\chapter}{plain}{empty}{}{}

%% for more attractive tables
\usepackage{booktabs}
\usepackage{longtable}


\usepackage{graphicx}


% Double spacing, if you want it.
\def\dsp{\def\baselinestretch{2.0}\large\normalsize}
% \dsp

% If the Grad. Division insists that the first paragraph of a section
% be indented (like the others), then include this line:
\usepackage{indentfirst}

%%%%%%%%%%%%%%%%%%
% If you want to use "sections" to partition your thesis
% un-comment the following:
%
% \counterwithout{section}{chapter}
% \setsecnumdepth{subsubsection}
% \def\sectionmark#1{\markboth{#1}{#1}}
% \def\subsectionmark#1{\markboth{#1}{#1}}
% \renewcommand{\thesection}{\arabic{section}}
% \renewcommand{\thesubsection}{\thesection.\arabic{subsection}}
% \makeatletter
% \let\l@subsection\l@section
% \let\l@section\l@chapter
% \makeatother

\renewcommand{\thetable}{\arabic{table}}
\renewcommand{\thefigure}{\arabic{figure}}

%%%%%%%%%%%%%%%%%%


%% Stuff from https://github.com/suchow/Dissertate

% The following line would print the thesis in a postscript font

% \usepackage{natbib}
% \def\bibpreamble{\protect\addcontentsline{toc}{chapter}{Bibliography}}

\setcounter{tocdepth}{1} % Print the chapter and sections to the toc
% controls depth of table of contents (toc): 0 = chapter, 1 = section, 2 = subsection

\usepackage{natbib}


% commands and environments needed by pandoc snippets
% extracted from the output of `pandoc -s`
%% Make R markdown code chunks work
\usepackage{array}
\usepackage{amssymb,amsmath}
\usepackage{ifxetex,ifluatex}
\ifxetex
  \usepackage{fontspec,xltxtra,xunicode}
  \defaultfontfeatures{Mapping=tex-text,Scale=MatchLowercase}
\else
  \ifluatex
    \usepackage{fontspec}
    \defaultfontfeatures{Mapping=tex-text,Scale=MatchLowercase}
  \else
    \usepackage[utf8]{inputenc}
  \fi
\fi
\usepackage{color}
\usepackage{fancyvrb}


\ifxetex
  \usepackage[setpagesize=false, % page size defined by xetex
              unicode=false, % unicode breaks when used with xetex
              xetex,
              colorlinks=true,
              linkcolor=blue]{hyperref}
\else
  \usepackage[unicode=true,
              colorlinks=true,
              linkcolor=blue]{hyperref}
\fi
\hypersetup{breaklinks=true, pdfborder={0 0 0}}
\setlength{\parindent}{20pt}
\setlength{\parskip}{6pt plus 2pt minus 1pt}
\setlength{\emergencystretch}{3em}  % prevent overfull lines
\setcounter{secnumdepth}{2} %% controls section numbering, e.g. 1 or 1.2, or 1.2.3



%  ----  Text Colors  ------------------------------------------
%
% Assign colors to writers for review
%
\newcommand{\authorcol}[1]{{\textcolor{blue}{#1}}}
\newcommand{\chaircol}[1]{{\textcolor{RedOrange}{#1}}}
\newcommand{\cochaircol}[1]{{\textcolor{Green}{#1}}}
%


%  --- Code chunk font size -----------------------------------------------
% https://stackoverflow.com/questions/38323331/code-chunk-font-size-in-beamer-with-knitr-and-latexhttps://stackoverflow.com/questions/38323331/code-chunk-font-size-in-beamer-with-knitr-and-latex

%% change fontsize of R code
\let\oldShaded\Shaded
\let\endoldShaded\endShaded
\renewenvironment{Shaded}{\footnotesize\oldShaded}{\endoldShaded}

%% change fontsize of output
\let\oldverbatim\verbatim
\let\endoldverbatim\endverbatim
\renewenvironment{verbatim}{\footnotesize\oldverbatim}{\endoldverbatim}


\begin{document}
% \linenumbers{}
%% Start formatting the first few special pages
%% frontmatter is needed to set the page numbering correctly
\frontmatter
%% from thesisdown
% To pass between YAML and LaTeX the dollar signs are added by CII
\title{DISSERTATION TITLE IN ALL CAPS}
\author{Author Full Name}
\adviser{Advisor name}
% \adviserAbstract{}
\major{Statistics}
\degreemonth{Month}
\degreeyear{Year}
% \copyrightpage
%%
%% For most people the defaults will be correct, so they are commented
%% out. To manually set these, just uncomment and make the needed
%% changes.
%% \college{Your college}
%% \city{Your City}
%%
%% For most people the following can be changed with a class
%% option. To manually set these, just uncomment the following and
%% make the needed changes.
%% \doctype{Thesis or Dissertation}
%% \degree{Your degree}
%% \degreeabbreviation{Your degree abbr.}
%%
%% Now that we know everything we need, we can generate the title page
%% itself.
%%
\maketitle


\begin{abstract}
    Here is my abstract. \emph{(350 word limit)}
\end{abstract}

%% Optional
\begin{copyrightpage}
\end{copyrightpage}

%% Optional
\begin{dedication}
Dedicated to\ldots{}
\end{dedication}

%%%%%%%%%%%%%%%%%%%
% Acknowledgments
%%%%%%%%%%%%%%%%%%%
\begin{acknowledgments}
Thank you to all my people!
\end{acknowledgments}
%%%%%%%%%%%%%%%%%%%

%%%%%%%%%%%%%%%%%%%
% Grant Information
%%%%%%%%%%%%%%%%%%%
% \begin{grantinfo}
%     % Add any grant info here
% \end{grantinfo}

%%%%%%%%%%%%%%%%%%%
% ToC
%%%%%%%%%%%%%%%%%%%
\tableofcontents

%%%%%%%%%%%%%%%%%%%
% List of Figures
%%%%%%%%%%%%%%%%%%%
\listoffigures
\listoftables

%%%%%%%%%%%%%%%%%%%
% Start of the document
%%%%%%%%%%%%%%%%%%%
\mainmatter


\hypertarget{introduction}{%
\chapter{Introduction}\label{introduction}}

Editing text colors: \authorcol{Author's editing color. Use \string\authorcol{}} Author may also use mostly black text as well. \chaircol{Chair's editing color. Use \string\chaircol{}} \cochaircol{Co-chair's editing color. Use \string\cochaircol{}} Redefine commands in unlthesis.cls with initials.

Welcome to the \emph{R Markdown} thesis template. This template is based on (and in many places copied directly from) the UW LaTeX template, and has been modified to use the UNL LaTeX thesis class maintained by the math department. Hopefully it will provide a nicer interface for those that have never used TeX or LaTeX before.

Using \emph{R Markdown} will also allow you to easily keep track of your analyses in \textbf{R} chunks of code, with the resulting plots and output included as well. The hope is this \emph{R Markdown} template gets you in the habit of doing reproducible research, which benefits you long-term as a researcher, but also will greatly help anyone that is trying to reproduce or build onto your results down the road.

Hopefully, you won't have much of a learning period to go through and you will reap the benefits of a nicely formatted thesis. The use of LaTeX in combination with \emph{Markdown} is more consistent than the output of a word processor, much less prone to corruption or crashing, and the resulting file is smaller than a Word file. While you may have never had problems using Word in the past, your thesis is likely going to be at least twice as large and complex as anything you've written before, taxing Word's capabilities. After working with \emph{Markdown} and \textbf{R} together for a few weeks, we are confident this will be your reporting style of choice going forward.

\textbf{Why use it?}

\emph{R Markdown} creates a simple and straightforward way to interface with the beauty of LaTeX. Packages have been written in \textbf{R} to work directly with LaTeX to produce nicely formatting tables and paragraphs. In addition to creating a user friendly interface to LaTeX, \emph{R Markdown} also allows you to read in your data, to analyze it and to visualize it using \textbf{R} functions, and also to provide the documentation and commentary on the results of your project. Further, it allows for \textbf{R} results to be passed inline to the commentary of your results. You'll see more on this later.

\textbf{Who should use it?}

Anyone who needs to use data analysis, math, tables, a lot of figures, complex cross-references, or who just cares about the final appearance of their document should use \emph{R Markdown}. Of particular use should be anyone in the sciences, but the user-friendly nature of \emph{Markdown} and its ability to keep track of and easily include figures, automatically generate a table of contents, index, references, table of figures, etc. should make it of great benefit to nearly anyone writing a thesis project.

\hypertarget{rmd-basics}{%
\chapter{R Markdown Basics}\label{rmd-basics}}

Here is a brief introduction into using \emph{R Markdown}. \emph{Markdown} is a simple formatting syntax for authoring HTML, PDF, and MS Word documents. \emph{R Markdown} provides the flexibility of \emph{Markdown} with the implementation of \textbf{R} input and output. For more details on using \emph{R Markdown} see \url{http://rmarkdown.rstudio.com}.

Be careful with your spacing in \emph{Markdown} documents. While whitespace largely is ignored, it does at times give \emph{Markdown} signals as to how to proceed. As a habit, try to keep everything left aligned whenever possible, especially as you type a new paragraph. In other words, there is no need to indent basic text in the Rmd document (in fact, it might cause your text to do funny things if you do).

\hypertarget{lists}{%
\section{Lists}\label{lists}}

It's easy to create a list. It can be unordered like

\begin{itemize}
\tightlist
\item
  Item 1
\item
  Item 2
\end{itemize}

or it can be ordered like

\begin{enumerate}
\def\labelenumi{\arabic{enumi}.}
\tightlist
\item
  Item 1
\item
  Item 2
\end{enumerate}

Notice that I intentionally mislabeled Item 2 as number 4. \emph{Markdown} automatically figures this out! You can put any numbers in the list and it will create the list. Check it out below.

To create a sublist, just indent the values a bit (at least four spaces or a tab). (Here's one case where indentation is key!)

\begin{enumerate}
\def\labelenumi{\arabic{enumi}.}
\tightlist
\item
  Item 1
\item
  Item 2
\item
  Item 3

  \begin{itemize}
  \tightlist
  \item
    Item 3a
  \item
    Item 3b
  \end{itemize}
\end{enumerate}

\hypertarget{line-breaks}{%
\section{Line breaks}\label{line-breaks}}

Make sure to add white space between lines if you'd like to start a new paragraph. Look at what happens below in the outputted document if you don't:

Here is the first sentence. Here is another sentence. Here is the last sentence to end the paragraph.
This should be a new paragraph.

\emph{Now for the correct way:}

Here is the first sentence. Here is another sentence. Here is the last sentence to end the paragraph.

This should be a new paragraph.

\hypertarget{r-chunks}{%
\section{R chunks}\label{r-chunks}}

When you click the \textbf{Knit} button above a document will be generated that includes both content as well as the output of any embedded \textbf{R} code chunks within the document. You can embed an \textbf{R} code chunk like this (\texttt{cars} is a built-in \textbf{R} dataset):

\begin{Shaded}
\begin{Highlighting}[]
\FunctionTok{summary}\NormalTok{(cars)}
\end{Highlighting}
\end{Shaded}

\begin{verbatim}
##      speed           dist       
##  Min.   : 4.0   Min.   :  2.00  
##  1st Qu.:12.0   1st Qu.: 26.00  
##  Median :15.0   Median : 36.00  
##  Mean   :15.4   Mean   : 42.98  
##  3rd Qu.:19.0   3rd Qu.: 56.00  
##  Max.   :25.0   Max.   :120.00
\end{verbatim}

\hypertarget{inline-code}{%
\section{Inline code}\label{inline-code}}

If you'd like to put the results of your analysis directly into your discussion, add inline code like this:

\begin{quote}
The \texttt{cos} of \(2 \pi\) is 1.
\end{quote}

Another example would be the direct calculation of the standard deviation:

\begin{quote}
The standard deviation of \texttt{speed} in \texttt{cars} is 5.2876444.
\end{quote}

One last neat feature is the use of the \texttt{ifelse} conditional statement which can be used to output text depending on the result of an \textbf{R} calculation:

\begin{quote}
The standard deviation is less than 6.
\end{quote}

Note the use of \texttt{\textgreater{}} here, which signifies a quotation environment that will be indented.

As you see with \texttt{\$2\ \textbackslash{}pi\$} above, mathematics can be added by surrounding the mathematical text with dollar signs. More examples of this are in \protect\hyperlink{math-sci}{Mathematics and Science} if you uncomment the code in \protect\hyperlink{math}{Math}.

\hypertarget{including-plots}{%
\section{Including plots}\label{including-plots}}

You can also embed plots. For example, here is a way to use the base \textbf{R} graphics package to produce a plot using the built-in \texttt{pressure} dataset:

\includegraphics[width=\linewidth]{thesis_files/figure-latex/pressure-1}

Note that the \texttt{echo=FALSE} parameter was added to the code chunk to prevent printing of the \textbf{R} code that generated the plot. There are plenty of other ways to add chunk options. More information is available at \url{http://yihui.name/knitr/options/}.

Another useful chunk option is the setting of \texttt{cache=TRUE} as you see here. If document rendering becomes time consuming due to long computations or plots that are expensive to generate you can use knitr caching to improve performance. Later in this file, you'll see a way to reference plots created in \textbf{R} or external figures.

\hypertarget{loading-and-exploring-data}{%
\section{Loading and exploring data}\label{loading-and-exploring-data}}

Included in this template is a file called \texttt{flights.csv}. This file includes a subset of the larger dataset of information about all flights that departed from Seattle and Portland in 2014. More information about this dataset and its \textbf{R} package is available at \url{http://github.com/ismayc/pnwflights14}. This subset includes only Portland flights and only rows that were complete with no missing values. Merges were also done with the \texttt{airports} and \texttt{airlines} data sets in the \texttt{pnwflights14} package to get more descriptive airport and airline names.

We can load in this data set using the following command:

\begin{Shaded}
\begin{Highlighting}[]
\NormalTok{flights }\OtherTok{\textless{}{-}} \FunctionTok{read.csv}\NormalTok{(}\StringTok{"data/flights.csv"}\NormalTok{)}
\end{Highlighting}
\end{Shaded}

The data is now stored in the data frame called \texttt{flights} in \textbf{R}. To get a better feel for the variables included in this dataset we can use a variety of functions. Here we can see the dimensions (rows by columns) and also the names of the columns.

\begin{Shaded}
\begin{Highlighting}[]
\FunctionTok{dim}\NormalTok{(flights)}
\end{Highlighting}
\end{Shaded}

\begin{verbatim}
## [1] 52808    16
\end{verbatim}

\begin{Shaded}
\begin{Highlighting}[]
\FunctionTok{names}\NormalTok{(flights)}
\end{Highlighting}
\end{Shaded}

\begin{verbatim}
##  [1] "month"        "day"          "dep_time"    
##  [4] "dep_delay"    "arr_time"     "arr_delay"   
##  [7] "carrier"      "tailnum"      "flight"      
## [10] "dest"         "air_time"     "distance"    
## [13] "hour"         "minute"       "carrier_name"
## [16] "dest_name"
\end{verbatim}

Another good idea is to take a look at the dataset in table form. With this dataset having more than 50,000 rows, we won't explicitly show the results of the command here. I recommend you enter the command into the Console \textbf{\emph{after}} you have run the \textbf{R} chunks above to load the data into \textbf{R}.

\begin{Shaded}
\begin{Highlighting}[]
\FunctionTok{View}\NormalTok{(flights)}
\end{Highlighting}
\end{Shaded}

While not required, it is highly recommended you use the \texttt{dplyr} package to manipulate and summarize your data set as needed. It uses a syntax that is easy to understand using chaining operations. Below I've created a few examples of using \texttt{dplyr} to get information about the Portland flights in 2014. You will also see the use of the \texttt{ggplot2} package, which produces beautiful, high-quality academic visuals.

We begin by checking to ensure that needed packages are installed and then we load them into our current working environment:

\begin{Shaded}
\begin{Highlighting}[]
\CommentTok{\# List of packages required for this}
\CommentTok{\# analysis}
\NormalTok{pkg }\OtherTok{\textless{}{-}} \FunctionTok{c}\NormalTok{(}\StringTok{"dplyr"}\NormalTok{, }\StringTok{"ggplot2"}\NormalTok{, }\StringTok{"knitr"}\NormalTok{, }\StringTok{"bookdown"}\NormalTok{,}
    \StringTok{"devtools"}\NormalTok{)}
\CommentTok{\# Check if packages are not installed and}
\CommentTok{\# assign the names of the packages not}
\CommentTok{\# installed to the variable new.pkg}
\NormalTok{new.pkg }\OtherTok{\textless{}{-}}\NormalTok{ pkg[}\SpecialCharTok{!}\NormalTok{(pkg }\SpecialCharTok{\%in\%} \FunctionTok{installed.packages}\NormalTok{())]}
\CommentTok{\# If there are any packages in the list that}
\CommentTok{\# aren\textquotesingle{}t installed, install them}
\ControlFlowTok{if}\NormalTok{ (}\FunctionTok{length}\NormalTok{(new.pkg)) }\FunctionTok{install.packages}\NormalTok{(new.pkg,}
    \AttributeTok{repos =} \StringTok{"http://cran.rstudio.com"}\NormalTok{)}
\CommentTok{\# Load packages (huskydown will load all of}
\CommentTok{\# the packages as well)}
\FunctionTok{library}\NormalTok{(huskydown)}
\end{Highlighting}
\end{Shaded}

\clearpage

The example we show here does the following:

\begin{itemize}
\item
  Selects only the \texttt{carrier\_name} and \texttt{arr\_delay} from the \texttt{flights} dataset and then assigns this subset to a new variable called \texttt{flights2}.
\item
  Using \texttt{flights2}, we determine the largest arrival delay for each of the carriers.
\end{itemize}

\begin{Shaded}
\begin{Highlighting}[]
\FunctionTok{library}\NormalTok{(dplyr)}
\end{Highlighting}
\end{Shaded}

\begin{verbatim}
## 
## Attaching package: 'dplyr'
\end{verbatim}

\begin{verbatim}
## The following objects are masked from 'package:stats':
## 
##     filter, lag
\end{verbatim}

\begin{verbatim}
## The following objects are masked from 'package:base':
## 
##     intersect, setdiff, setequal, union
\end{verbatim}

\begin{Shaded}
\begin{Highlighting}[]
\NormalTok{flights2 }\OtherTok{\textless{}{-}}\NormalTok{ flights }\SpecialCharTok{\%\textgreater{}\%}
    \FunctionTok{select}\NormalTok{(carrier\_name, arr\_delay)}
\NormalTok{max\_delays }\OtherTok{\textless{}{-}}\NormalTok{ flights2 }\SpecialCharTok{\%\textgreater{}\%}
    \FunctionTok{group\_by}\NormalTok{(carrier\_name) }\SpecialCharTok{\%\textgreater{}\%}
    \FunctionTok{summarize}\NormalTok{(}\AttributeTok{max\_arr\_delay =} \FunctionTok{max}\NormalTok{(arr\_delay, }\AttributeTok{na.rm =} \ConstantTok{TRUE}\NormalTok{))}
\end{Highlighting}
\end{Shaded}

A useful function in the \texttt{knitr} package for making nice tables in \emph{R Markdown} is called \texttt{kable}. It is much easier to use than manually entering values into a table by copying and pasting values into Excel or LaTeX. This again goes to show how nice reproducible documents can be! (Note the use of \texttt{results="asis"}, which will produce the table instead of the code to create the table.) The \texttt{caption.short} argument is used to include a shorter title to appear in the List of Tables.

\begin{Shaded}
\begin{Highlighting}[]
\FunctionTok{library}\NormalTok{(knitr)}
\FunctionTok{kable}\NormalTok{(max\_delays, }\AttributeTok{col.names =} \FunctionTok{c}\NormalTok{(}\StringTok{"Airline"}\NormalTok{, }\StringTok{"Max Arrival Delay"}\NormalTok{),}
    \AttributeTok{caption =} \StringTok{"Maximum Delays by Airline"}\NormalTok{, }\AttributeTok{caption.short =} \StringTok{"Max Delays by Airline"}\NormalTok{,}
    \AttributeTok{longtable =} \ConstantTok{TRUE}\NormalTok{, }\AttributeTok{booktabs =} \ConstantTok{TRUE}\NormalTok{)}
\end{Highlighting}
\end{Shaded}

\begin{longtable}[t]{lr}
\caption[Max Delays by Airline]{\label{tab:maxdelays}Maximum Delays by Airline}\\
\toprule
Airline & Max Arrival Delay\\
\midrule
Alaska Airlines Inc. & 338\\
American Airlines Inc. & 1539\\
Delta Air Lines Inc. & 651\\
Frontier Airlines Inc. & 575\\
Hawaiian Airlines Inc. & 407\\
\addlinespace
JetBlue Airways & 273\\
SkyWest Airlines Inc. & 421\\
Southwest Airlines Co. & 694\\
United Air Lines Inc. & 472\\
US Airways Inc. & 347\\
\addlinespace
Virgin America & 366\\
\bottomrule
\end{longtable}

The last two options make the table a little easier-to-read.

We can further look into the properties of the largest value here for American Airlines Inc.~To do so, we can isolate the row corresponding to the arrival delay of 1539 minutes for American in our original \texttt{flights} dataset.

\begin{Shaded}
\begin{Highlighting}[]
\NormalTok{flights }\SpecialCharTok{\%\textgreater{}\%}
    \FunctionTok{filter}\NormalTok{(arr\_delay }\SpecialCharTok{==} \DecValTok{1539}\NormalTok{, carrier\_name }\SpecialCharTok{==}
        \StringTok{"American Airlines Inc."}\NormalTok{) }\SpecialCharTok{\%\textgreater{}\%}
    \FunctionTok{select}\NormalTok{(}\SpecialCharTok{{-}}\FunctionTok{c}\NormalTok{(month, day, carrier, dest\_name,}
\NormalTok{        hour, minute, carrier\_name, arr\_delay))}
\end{Highlighting}
\end{Shaded}

\begin{verbatim}
##   dep_time dep_delay arr_time tailnum flight dest air_time
## 1     1403      1553     1934  N595AA   1568  DFW      182
##   distance
## 1     1616
\end{verbatim}

We see that the flight occurred on March 3rd and departed a little after 2 PM on its way to Dallas/Fort Worth. Lastly, we show how we can visualize the arrival delay of all departing flights from Portland on March 3rd against time of departure.

\begin{Shaded}
\begin{Highlighting}[]
\FunctionTok{library}\NormalTok{(ggplot2)}
\NormalTok{flights }\SpecialCharTok{\%\textgreater{}\%}
    \FunctionTok{filter}\NormalTok{(month }\SpecialCharTok{==} \DecValTok{3}\NormalTok{, day }\SpecialCharTok{==} \DecValTok{3}\NormalTok{) }\SpecialCharTok{\%\textgreater{}\%}
    \FunctionTok{ggplot}\NormalTok{(}\FunctionTok{aes}\NormalTok{(}\AttributeTok{x =}\NormalTok{ dep\_time, }\AttributeTok{y =}\NormalTok{ arr\_delay)) }\SpecialCharTok{+}
    \FunctionTok{geom\_point}\NormalTok{()}
\end{Highlighting}
\end{Shaded}

\includegraphics[width=\linewidth]{thesis_files/figure-latex/march3plot-1}

\hypertarget{additional-resources}{%
\section{Additional resources}\label{additional-resources}}

\begin{itemize}
\item
  \emph{Markdown} Cheatsheet - \url{https://github.com/adam-p/markdown-here/wiki/Markdown-Cheatsheet}
\item
  \emph{R Markdown} Reference Guide - \url{https://www.rstudio.com/wp-content/uploads/2015/03/rmarkdown-reference.pdf}
\item
  Introduction to \texttt{dplyr} - \url{https://cran.rstudio.com/web/packages/dplyr/vignettes/introduction.html}
\item
  \texttt{ggplot2} Documentation - \url{http://docs.ggplot2.org/current/}
\end{itemize}

\hypertarget{math-sci}{%
\chapter{Mathematics and Science}\label{math-sci}}

\hypertarget{math}{%
\section{Math}\label{math}}

\TeX~is the best way to typeset mathematics. Donald Knuth designed \TeX~when he got frustrated at how long it was taking the typesetters to finish his book, which contained a lot of mathematics. One nice feature of \emph{R Markdown} is its ability to read LaTeX code directly.

If you are doing a thesis that will involve lots of math, you will want to read the following section which has been commented out. If you're not going to use math, skip over or delete this next commented section.

\hypertarget{chemistry-101-symbols}{%
\section{Chemistry 101: Symbols}\label{chemistry-101-symbols}}

Chemical formulas will look best if they are not italicized. Get around math mode's automatic italicizing in LaTeX by using the argument \texttt{\$\textbackslash{}mathrm\{formula\ here\}\$}, with your formula inside the curly brackets. (Notice the use of the backticks here which enclose text that acts as code.)

So, \(\mathrm{Fe_2^{2+}Cr_2O_4}\) is written \texttt{\$\textbackslash{}mathrm\{Fe\_2\^{}\{2+\}Cr\_2O\_4\}\$}.

\noindent Exponent or Superscript: \(\mathrm{O^-}\)

\noindent Subscript: \(\mathrm{CH_4}\)

To stack numbers or letters as in \(\mathrm{Fe_2^{2+}}\), the subscript is defined first, and then the superscript is defined.

\noindent Bullet: CuCl \(\bullet\) \(\mathrm{7H_{2}O}\)

\noindent Delta: \(\Delta\)

\noindent Reaction Arrows: \(\longrightarrow\) or \(\xrightarrow{solution}\)

\noindent Resonance Arrows: \(\leftrightarrow\)

\noindent Reversible Reaction Arrows: \(\rightleftharpoons\)

\hypertarget{typesetting-reactions}{%
\subsection{Typesetting reactions}\label{typesetting-reactions}}

You may wish to put your reaction in an equation environment, which means that LaTeX will place the reaction where it fits and will number the equations for you.

\begin{equation}
  \mathrm{C_6H_{12}O_6  + 6O_2} \longrightarrow \mathrm{6CO_2 + 6H_2O}
  \label{eq:reaction}
\end{equation}

We can reference this combustion of glucose reaction via Equation \eqref{eq:reaction}.

\hypertarget{other-examples-of-reactions}{%
\subsection{Other examples of reactions}\label{other-examples-of-reactions}}

\(\mathrm{NH_4Cl_{(s)}}\) \(\rightleftharpoons\) \(\mathrm{NH_{3(g)}+HCl_{(g)}}\)

\noindent \(\mathrm{MeCH_2Br + Mg}\) \(\xrightarrow[below]{above}\) \(\mathrm{MeCH_2\bullet Mg \bullet Br}\)

\hypertarget{physics}{%
\section{Physics}\label{physics}}

Many of the symbols you will need can be found on the math page \url{http://web.reed.edu/cis/help/latex/math.html} and the Comprehensive LaTeX Symbol Guide (\url{http://mirror.utexas.edu/ctan/info/symbols/comprehensive/symbols-letter.pdf}).

\hypertarget{biology}{%
\section{Biology}\label{biology}}

You will probably find the resources at \url{http://www.lecb.ncifcrf.gov/~toms/latex.html} helpful, particularly the links to bsts for various journals. You may also be interested in TeXShade for nucleotide typesetting (\url{http://homepages.uni-tuebingen.de/beitz/txe.html}). Be sure to read the proceeding chapter on graphics and tables.

\hypertarget{ref-labels}{%
\chapter{Tables, Graphics, References, and Labels}\label{ref-labels}}

\hypertarget{tables}{%
\section{Tables}\label{tables}}

By far the easiest way to present tables in your thesis is to store the contents of the table in a CSV or Excel file, then read that file in to your R Markdown document as a data frame. Then you can style the table with the \texttt{kable} function, or functions in the \href{https://cran.r-project.org/web/packages/kableExtra/index.html}{kableExtra} pacakge.

In addition to the tables that can be automatically generated from a data frame in \textbf{R} that you saw in \protect\hyperlink{rmd-basics}{R Markdown Basics} using the \texttt{kable} function, you can also create tables using \emph{pandoc}. (More information is available at \url{http://pandoc.org/README.html\#tables}.) This might be useful if you don't have values specifically stored in \textbf{R}, but you'd like to display them in table form. Below is an example. Pay careful attention to the alignment in the table and hyphens to create the rows and columns. Generally I don't recommend this approach of typing the table directly into your R Markdown document.

\begin{longtable}[]{@{}
  >{\centering\arraybackslash}p{(\columnwidth - 4\tabcolsep) * \real{0.3133}}
  >{\centering\arraybackslash}p{(\columnwidth - 4\tabcolsep) * \real{0.5060}}
  >{\centering\arraybackslash}p{(\columnwidth - 4\tabcolsep) * \real{0.1807}}@{}}
\caption{\label{tab:inher} Correlation of Inheritance Factors for Parents and Child}\tabularnewline
\toprule
\begin{minipage}[b]{\linewidth}\centering
Factors
\end{minipage} & \begin{minipage}[b]{\linewidth}\centering
Correlation between Parents \& Child
\end{minipage} & \begin{minipage}[b]{\linewidth}\centering
Inherited
\end{minipage} \\
\midrule
\endfirsthead
\toprule
\begin{minipage}[b]{\linewidth}\centering
Factors
\end{minipage} & \begin{minipage}[b]{\linewidth}\centering
Correlation between Parents \& Child
\end{minipage} & \begin{minipage}[b]{\linewidth}\centering
Inherited
\end{minipage} \\
\midrule
\endhead
Education & -0.49 & Yes \\
Socio-Economic Status & 0.28 & Slight \\
Income & 0.08 & No \\
Family Size & 0.18 & Slight \\
Occupational Prestige & 0.21 & Slight \\
\bottomrule
\end{longtable}

We can also create a link to the table by doing the following: Table \ref{tab:inher}. If you go back to \protect\hyperlink{loading-and-exploring-data}{Loading and exploring data} and look at the \texttt{kable} table, we can create a reference to this max delays table too: Table \ref{tab:maxdelays}. The addition of the \texttt{(\textbackslash{}\#tab:inher)} option to the end of the table caption allows us to then make a reference to Table \texttt{\textbackslash{}@ref(tab:label)}. Note that this reference could appear anywhere throughout the document after the table has appeared.

\clearpage

\hypertarget{figures}{%
\section{Figures}\label{figures}}

If your thesis has a lot of figures, \emph{R Markdown} might behave better for you than that other word processor. One perk is that it will automatically number the figures accordingly in each chapter. You'll also be able to create a label for each figure, add a caption, and then reference the figure in a way similar to what we saw with tables earlier. If you label your figures, you can move the figures around and \emph{R Markdown} will automatically adjust the numbering for you. No need for you to remember! So that you don't have to get too far into LaTeX to do this, a couple \textbf{R} functions have been created for you to assist. You'll see their use below.

In the \textbf{R} chunk below, we will load in a picture stored as \texttt{uw.png} in our main directory. We then give it the caption of ``UW logo'', the label of ``uwlogo'', and specify that this is a figure. Make note of the different \textbf{R} chunk options that are given in the R Markdown file (not shown in the knitted document).

\begin{Shaded}
\begin{Highlighting}[]
\FunctionTok{include\_graphics}\NormalTok{(}\AttributeTok{path =} \StringTok{"figure/unl.png"}\NormalTok{)}
\end{Highlighting}
\end{Shaded}

\begin{figure}
\includegraphics[width=\linewidth]{figure/unl} \caption{logo}\label{fig:uwlogo}
\end{figure}

Here is a reference to the UW logo: Figure \ref{fig:uwlogo}. Note the use of the \texttt{fig:} code here. By naming the \textbf{R} chunk that contains the figure, we can then reference that figure later as done in the first sentence here. We can also specify the caption for the figure via the R chunk option \texttt{fig.cap}.

\clearpage

Below we will investigate how to save the output of an \textbf{R} plot and label it in a way similar to that done above. Recall the \texttt{flights} dataset from Chapter \ref{rmd-basics}. (Note that we've shown a different way to reference a section or chapter here.) We will next explore a bar graph with the mean flight departure delays by airline from Portland for 2014. Note also the use of the \texttt{scale} parameter which is discussed on the next page.

\begin{Shaded}
\begin{Highlighting}[]
\NormalTok{flights }\SpecialCharTok{\%\textgreater{}\%}
    \FunctionTok{group\_by}\NormalTok{(carrier) }\SpecialCharTok{\%\textgreater{}\%}
    \FunctionTok{summarize}\NormalTok{(}\AttributeTok{mean\_dep\_delay =} \FunctionTok{mean}\NormalTok{(dep\_delay)) }\SpecialCharTok{\%\textgreater{}\%}
    \FunctionTok{ggplot}\NormalTok{(}\FunctionTok{aes}\NormalTok{(}\AttributeTok{x =}\NormalTok{ carrier, }\AttributeTok{y =}\NormalTok{ mean\_dep\_delay)) }\SpecialCharTok{+}
    \FunctionTok{geom\_bar}\NormalTok{(}\AttributeTok{position =} \StringTok{"identity"}\NormalTok{, }\AttributeTok{stat =} \StringTok{"identity"}\NormalTok{,}
        \AttributeTok{fill =} \StringTok{"red"}\NormalTok{)}
\end{Highlighting}
\end{Shaded}

\begin{figure}
\includegraphics[width=\linewidth]{thesis_files/figure-latex/delaysboxplot-1} \caption{Mean Delays by Airline}\label{fig:delaysboxplot}
\end{figure}

Here is a reference to this image: Figure \ref{fig:delaysboxplot}.

A table linking these carrier codes to airline names is available at \url{https://github.com/ismayc/pnwflights14/blob/master/data/airlines.csv}.

\clearpage

Next, we will explore the use of the \texttt{out.extra} chunk option, which can be used to shrink or expand an image loaded from a file by specifying \texttt{"scale=\ "}. Here we use the mathematical graph stored in the ``subdivision.pdf'' file.

\begin{figure}
\includegraphics[width=\linewidth,scale=0.75]{figure/subdivision} \caption{Subdiv. graph}\label{fig:subd}
\end{figure}

Here is a reference to this image: Figure \ref{fig:subd}. Note that \texttt{echo=FALSE} is specified so that the \textbf{R} code is hidden in the document.

\textbf{More Figure Stuff}

Lastly, we will explore how to rotate and enlarge figures using the \texttt{out.extra} chunk option. (Currently this only works in the PDF version of the book.)

\begin{figure}
\includegraphics[width=\linewidth,angle=180, scale=1.1]{figure/subdivision} \caption{A Larger Figure, Flipped Upside Down}\label{fig:subd2}
\end{figure}

As another example, here is a reference: Figure \ref{fig:subd2}.

\hypertarget{footnotes-and-endnotes}{%
\section{Footnotes and Endnotes}\label{footnotes-and-endnotes}}

You might want to footnote something. \footnote{footnote text} The footnote will be in a smaller font and placed appropriately. Endnotes work in much the same way.

\hypertarget{cross-referencing-chapters-and-sections}{%
\section{Cross-referencing chapters and sections}\label{cross-referencing-chapters-and-sections}}

The \href{https://bookdown.org/yihui/bookdown/cross-references.html}{bookdown documentation} is an excellent source for learning how to cross-reference in a bookdown project such as a huskydown document. Here we only cover the most common uses for a typical thesis. If you want something more complex or fancy, please refer to the bookdown documentation and seek help from the developers of that package.

By default, all of your chapter and section headers will get an auto-generated ID label For example, e.g., \texttt{\#\ Chapter\ 1} will have an auto-generated ID \texttt{chapter-1}. Note that the ID label is all lower case, and has no spaces. If you have any kind of punctuation in your header, such as a colon (:), it will not appear in the ID label. Then in your text you can reference chapter one in your Rmd file like this: `as discussed in Chapter \texttt{\textbackslash{}@ref(chapter-1)}', which will print as `as discussed in Chapter 1'

We strongly recommend that you to manually assign ID labels to your chapter header to make it easy to cross-reference. For example, at the top of the Rmd file for this chapter, you can see:

\texttt{\#\ Tables,\ Graphics,\ References,\ and\ Labels\ \{\#ref-labels\}}

The \texttt{\{\#ref-labels\}} part of this header is the ID label. It doesn't show in the output, but is there for us to use for easy cross-referencing, because it can be short, and we don't need to change it elsewhere our document when we update the chapter header. We can use this custom ID label in our Rmd document like this: `as discussed in Chapter \texttt{\textbackslash{}@ref(ref-labels)}', which will print as `as discussed in Chapter \ref{ref-labels}'. If you need to show custom text instead of the chapter number, you use this syntax in your Rmd document: \texttt{see\ {[}my\ chapter\ about\ labels{]}(\#ref-labels)\ for\ more\ details} which will appear as `see \protect\hyperlink{ref-labels}{my chapter about labels} for more details'

To cross-reference a specific section in the same chapter, we recommend adding a custom ID label to the section header, and using that to cross-reference. For example, earlier in this chapter we have a section on tables and in the Rmd file we see \texttt{\#\#\ Tables\ \{\#tables\}}. We can cross-reference that in the text like this `as discussed in the section on \texttt{{[}tables{]}(\#tables)}' which will appear as `as discussed in the above section on \protect\hyperlink{tables}{tables}'

To cross-reference a section in a different chapter we can use the ID label from that section directly. For example, we can write in our Rmd document \texttt{as\ discussed\ in\ the\ section\ on\ {[}R\ code\ chunks{]}(\#r-chunks)\ in\ Chapter\ \textbackslash{}@ref(rmd-basics)} which will appear as `as discussed in the section on \protect\hyperlink{r-chunks}{R code chunks} in Chapter \ref{rmd-basics}'.

If you prefer to cross-reference by the section number, we can use custom ID labels in our Rmd document. For example, to refer to a section in our first chapter, we can write in the Rmd document: \texttt{as\ discussed\ in\ section\ \textbackslash{}@ref(r-chunks)\ in\ Chapter\ \textbackslash{}@ref(rmd-basics)}. This will appear with section and chapter numbers like so: as `as discussed in section \ref{r-chunks} in Chapter \ref{rmd-basics}'.

\hypertarget{bibliographies}{%
\section{Bibliographies}\label{bibliographies}}

Of course you will need to cite things, and you will probably accumulate an armful of sources. There are a variety of tools available for creating a bibliography database (stored with the .bib extension). In addition to BibTeX suggested below, you may want to consider using the free and easy-to-use tool called Zotero. Some Zotero documentation is at \url{http://libguides.reed.edu/citation/zotero}. In addition, a tutorial is available from Middlebury College at \url{http://sites.middlebury.edu/zoteromiddlebury/}.

\emph{R Markdown} uses \emph{pandoc} (\url{http://pandoc.org/}) to build its bibliographies. One nice caveat of this is that you won't have to do a second compile to load in references as standard LaTeX requires. To cite references in your thesis (after creating your bibliography database), place the reference name inside square brackets and precede it by the ``at'' symbol. For example, here's a reference to a book about worrying: (Molina \& Borkovec, 1994). This \texttt{Molina1994} entry appears in a file called \texttt{thesis.bib} in the \texttt{bib} folder. This bibliography database file was created by a program called BibTeX. You can call this file something else if you like (look at the YAML header in the main .Rmd file) and, by default, is to placed in the \texttt{bib} folder.

For more information about BibTeX and bibliographies, see (\url{http://web.reed.edu/cis/help/latex/index.html})\footnote{Reed~College (2007)}. There are three pages on this topic: \emph{bibtex} (which talks about using BibTeX, at \url{http://web.reed.edu/cis/help/latex/bibtex.html}), \emph{bibtexstyles} (about how to find and use the bibliography style that best suits your needs, at \url{http://web.reed.edu/cis/help/latex/bibtexstyles.html}) and \emph{bibman} (which covers how to make and maintain a bibliography by hand, without BibTeX, at \url{http://web.reed.edu/cis/help/latex/bibman.html}). The last page will not be useful unless you have only a few sources.

If you look at the YAML header at the top of the main .Rmd file you can see that we can specify the style of the bibliography by referencing the appropriate csl file. You can download a variety of different style files at \url{https://www.zotero.org/styles}. Make sure to download the file into the csl folder.

\textbf{Tips for Bibliographies}

\begin{itemize}
\tightlist
\item
  Like with thesis formatting, the sooner you start compiling your bibliography for something as large as thesis, the better.
\item
  The cite key (a citation's label) needs to be unique from the other entries.
\item
  When you have more than one author or editor, you need to separate each author's name by the word ``and'' e.g.~\texttt{Author\ =\ \{Noble,\ Sam\ and\ Youngberg,\ Jessica\},}.
\item
  Bibliographies made using BibTeX (whether manually or using a manager) accept LaTeX markup, so you can italicize and add symbols as necessary.
\item
  To force capitalization in an article title or where all lowercase is generally used, bracket the capital letter in curly braces.
\end{itemize}

\hypertarget{anything-else}{%
\section{Anything else?}\label{anything-else}}

If you'd like to see examples of other things in this template, please \href{https://github.com/benmarwick/huskydown/issues/new}{contact us} (email \href{mailto:bmarwick@uw.edu}{\nolinkurl{bmarwick@uw.edu}}) with your suggestions. We love to see people using \emph{R Markdown} for their theses, and are happy to help.

\hypertarget{conclusion}{%
\chapter*{Conclusion}\label{conclusion}}
\addcontentsline{toc}{chapter}{Conclusion}

If we don't want Conclusion to have a chapter number next to it, we can add the \texttt{\{-\}} attribute.

\textbf{More info}

And here's some other random info: the first paragraph after a chapter title or section head \emph{shouldn't be} indented, because indents are to tell the reader that you're starting a new paragraph. Since that's obvious after a chapter or section title, proper typesetting doesn't add an indent there.

\appendix

\hypertarget{the-first-appendix}{%
\chapter{The First Appendix}\label{the-first-appendix}}

This first appendix includes all of the R chunks of code that were hidden throughout the document (using the \texttt{include\ =\ FALSE} chunk tag) to help with readibility and/or setup.

\textbf{In the main Rmd file}

\begin{Shaded}
\begin{Highlighting}[]
\CommentTok{\# This chunk ensures that the huskydown}
\CommentTok{\# package is installed and loaded. This}
\CommentTok{\# huskydown package includes the template}
\CommentTok{\# files for the thesis.}
\ControlFlowTok{if}\NormalTok{ (}\SpecialCharTok{!}\FunctionTok{require}\NormalTok{(devtools)) }\FunctionTok{install.packages}\NormalTok{(}\StringTok{"devtools"}\NormalTok{,}
    \AttributeTok{repos =} \StringTok{"http://cran.rstudio.com"}\NormalTok{)}
\ControlFlowTok{if}\NormalTok{ (}\SpecialCharTok{!}\FunctionTok{require}\NormalTok{(huskydown)) devtools}\SpecialCharTok{::}\FunctionTok{install\_github}\NormalTok{(}\StringTok{"benmarwick/huskydown"}\NormalTok{)}
\FunctionTok{library}\NormalTok{(huskydown)}
\end{Highlighting}
\end{Shaded}

\textbf{In Chapter \ref{ref-labels}:}

\begin{Shaded}
\begin{Highlighting}[]
\CommentTok{\# This chunk ensures that the huskydown}
\CommentTok{\# package is installed and loaded. This}
\CommentTok{\# huskydown package includes the template}
\CommentTok{\# files for the thesis and also two}
\CommentTok{\# functions used for labeling and}
\CommentTok{\# referencing}
\ControlFlowTok{if}\NormalTok{ (}\SpecialCharTok{!}\FunctionTok{require}\NormalTok{(devtools)) }\FunctionTok{install.packages}\NormalTok{(}\StringTok{"devtools"}\NormalTok{,}
    \AttributeTok{repos =} \StringTok{"http://cran.rstudio.com"}\NormalTok{)}
\ControlFlowTok{if}\NormalTok{ (}\SpecialCharTok{!}\FunctionTok{require}\NormalTok{(dplyr)) }\FunctionTok{install.packages}\NormalTok{(}\StringTok{"dplyr"}\NormalTok{,}
    \AttributeTok{repos =} \StringTok{"http://cran.rstudio.com"}\NormalTok{)}
\ControlFlowTok{if}\NormalTok{ (}\SpecialCharTok{!}\FunctionTok{require}\NormalTok{(ggplot2)) }\FunctionTok{install.packages}\NormalTok{(}\StringTok{"ggplot2"}\NormalTok{,}
    \AttributeTok{repos =} \StringTok{"http://cran.rstudio.com"}\NormalTok{)}
\ControlFlowTok{if}\NormalTok{ (}\SpecialCharTok{!}\FunctionTok{require}\NormalTok{(ggplot2)) }\FunctionTok{install.packages}\NormalTok{(}\StringTok{"bookdown"}\NormalTok{,}
    \AttributeTok{repos =} \StringTok{"http://cran.rstudio.com"}\NormalTok{)}
\ControlFlowTok{if}\NormalTok{ (}\SpecialCharTok{!}\FunctionTok{require}\NormalTok{(huskydown)) \{}
    \FunctionTok{library}\NormalTok{(devtools)}
\NormalTok{    devtools}\SpecialCharTok{::}\FunctionTok{install\_github}\NormalTok{(}\StringTok{"benmarwick/huskydown"}\NormalTok{)}
\NormalTok{\}}
\FunctionTok{library}\NormalTok{(huskydown)}
\NormalTok{flights }\OtherTok{\textless{}{-}} \FunctionTok{read.csv}\NormalTok{(}\StringTok{"data/flights.csv"}\NormalTok{)}
\end{Highlighting}
\end{Shaded}

\hypertarget{the-second-appendix-for-fun}{%
\chapter{The Second Appendix, for Fun}\label{the-second-appendix-for-fun}}

\hypertarget{colophon}{%
\chapter*{Colophon}\label{colophon}}
\addcontentsline{toc}{chapter}{Colophon}

This document is set in \href{https://github.com/georgd/EB-Garamond}{EB Garamond}, \href{https://github.com/adobe-fonts/source-code-pro/}{Source Code Pro} and \href{http://www.latofonts.com/lato-free-fonts/}{Lato}. The body text is set at 11pt with \(\familydefault\).

It was written in R Markdown and \(\LaTeX\), and rendered into PDF using \href{https://github.com/benmarwick/huskydown}{huskydown} and \href{https://github.com/rstudio/bookdown}{bookdown}.

This document was typeset using the XeTeX typesetting system, and the \href{http://staff.washington.edu/fox/tex/}{University of Washington Thesis class} class created by Jim Fox. Under the hood, the \href{https://github.com/UWIT-IAM/UWThesis}{University of Washington Thesis LaTeX template} is used to ensure that documents conform precisely to submission standards. Other elements of the document formatting source code have been taken from the \href{https://github.com/stevenpollack/ucbthesis}{Latex, Knitr, and RMarkdown templates for UC Berkeley's graduate thesis}, and \href{https://github.com/suchow/Dissertate}{Dissertate: a LaTeX dissertation template to support the production and typesetting of a PhD dissertation at Harvard, Princeton, and NYU}

The source files for this thesis, along with all the data files, have been organised into an R package, xxx, which is available at \url{https://github.com/xxx/xxx}. A hard copy of the thesis can be found in the University of Washington library.

This version of the thesis was generated on 2022-06-04 14:57:50. The repository is currently at this commit:

The computational environment that was used to generate this version is as follows:

\begin{verbatim}
## - Session info -------------------------------------------
##  setting  value
##  version  R version 4.1.2 (2021-11-01)
##  os       Windows 10 x64 (build 22000)
##  system   x86_64, mingw32
##  ui       RTerm
##  language (EN)
##  collate  English_United States.1252
##  ctype    English_United States.1252
##  tz       America/Chicago
##  date     2022-06-04
##  pandoc   2.17.0.1 @ C:/Users/ERobi/AppData/Local/Programs/Quarto/bin/ (via rmarkdown)
## 
## - Packages -----------------------------------------------
##  package     * version date (UTC) lib source
##  assertthat    0.2.1   2019-03-21 [1] CRAN (R 4.1.2)
##  bookdown      0.24.10 2022-03-09 [1] Github (rstudio/bookdown@2821273)
##  cachem        1.0.6   2021-08-19 [1] CRAN (R 4.1.2)
##  callr         3.7.0   2021-04-20 [1] CRAN (R 4.1.2)
##  cli           3.1.0   2021-10-27 [1] CRAN (R 4.1.2)
##  colorspace    2.0-3   2022-02-21 [1] CRAN (R 4.1.2)
##  crayon        1.5.0   2022-02-14 [1] CRAN (R 4.1.2)
##  DBI           1.1.2   2021-12-20 [1] CRAN (R 4.1.2)
##  desc          1.4.1   2022-03-06 [1] CRAN (R 4.1.2)
##  devtools    * 2.4.3   2021-11-30 [1] CRAN (R 4.1.2)
##  digest        0.6.29  2021-12-01 [1] CRAN (R 4.1.2)
##  dplyr       * 1.0.8   2022-02-08 [1] CRAN (R 4.1.2)
##  ellipsis      0.3.2   2021-04-29 [1] CRAN (R 4.1.2)
##  evaluate      0.15    2022-02-18 [1] CRAN (R 4.1.2)
##  fansi         1.0.2   2022-01-14 [1] CRAN (R 4.1.2)
##  farver        2.1.0   2021-02-28 [1] CRAN (R 4.1.2)
##  fastmap       1.1.0   2021-01-25 [1] CRAN (R 4.1.2)
##  formatR       1.12    2022-03-31 [1] CRAN (R 4.1.3)
##  fs            1.5.2   2021-12-08 [1] CRAN (R 4.1.2)
##  generics      0.1.2   2022-01-31 [1] CRAN (R 4.1.2)
##  ggplot2     * 3.3.5   2021-06-25 [1] CRAN (R 4.1.2)
##  git2r         0.29.0  2021-11-22 [1] CRAN (R 4.1.2)
##  glue          1.6.0   2021-12-17 [1] CRAN (R 4.1.2)
##  gtable        0.3.0   2019-03-25 [1] CRAN (R 4.1.2)
##  htmltools     0.5.2   2021-08-25 [1] CRAN (R 4.1.2)
##  huskydown   * 0.0.5   2022-03-09 [1] Github (benmarwick/huskydown@addb48e)
##  knitr       * 1.37    2021-12-16 [1] CRAN (R 4.1.2)
##  labeling      0.4.2   2020-10-20 [1] CRAN (R 4.1.1)
##  lifecycle     1.0.1   2021-09-24 [1] CRAN (R 4.1.2)
##  magrittr      2.0.1   2020-11-17 [1] CRAN (R 4.1.2)
##  memoise       2.0.1   2021-11-26 [1] CRAN (R 4.1.2)
##  munsell       0.5.0   2018-06-12 [1] CRAN (R 4.1.2)
##  pillar        1.7.0   2022-02-01 [1] CRAN (R 4.1.2)
##  pkgbuild      1.3.1   2021-12-20 [1] CRAN (R 4.1.2)
##  pkgconfig     2.0.3   2019-09-22 [1] CRAN (R 4.1.2)
##  pkgload       1.2.4   2021-11-30 [1] CRAN (R 4.1.2)
##  prettyunits   1.1.1   2020-01-24 [1] CRAN (R 4.1.2)
##  processx      3.5.2   2021-04-30 [1] CRAN (R 4.1.2)
##  ps            1.6.0   2021-02-28 [1] CRAN (R 4.1.2)
##  purrr         0.3.4   2020-04-17 [1] CRAN (R 4.1.2)
##  R6            2.5.1   2021-08-19 [1] CRAN (R 4.1.2)
##  remotes       2.4.2   2021-11-30 [1] CRAN (R 4.1.2)
##  rlang         1.0.2   2022-03-04 [1] CRAN (R 4.1.2)
##  rmarkdown     2.14    2022-04-25 [1] CRAN (R 4.1.3)
##  rprojroot     2.0.2   2020-11-15 [1] CRAN (R 4.1.2)
##  rstudioapi    0.13    2020-11-12 [1] CRAN (R 4.1.2)
##  scales        1.1.1   2020-05-11 [1] CRAN (R 4.1.2)
##  sessioninfo   1.2.2   2021-12-06 [1] CRAN (R 4.1.2)
##  stringi       1.7.6   2021-11-29 [1] CRAN (R 4.1.2)
##  stringr       1.4.0   2019-02-10 [1] CRAN (R 4.1.2)
##  testthat      3.1.1   2021-12-03 [1] CRAN (R 4.1.2)
##  tibble        3.1.6   2021-11-07 [1] CRAN (R 4.1.2)
##  tidyselect    1.1.2   2022-02-21 [1] CRAN (R 4.1.2)
##  usethis     * 2.1.5   2021-12-09 [1] CRAN (R 4.1.2)
##  utf8          1.2.2   2021-07-24 [1] CRAN (R 4.1.2)
##  vctrs         0.3.8   2021-04-29 [1] CRAN (R 4.1.2)
##  withr         2.5.0   2022-03-03 [1] CRAN (R 4.1.2)
##  xfun          0.29    2021-12-14 [1] CRAN (R 4.1.2)
##  yaml          2.2.1   2020-02-01 [1] CRAN (R 4.1.1)
## 
##  [1] C:/Users/ERobi/Documents/R/win-library/4.1
##  [2] C:/Program Files/R/R-4.1.2/library
## 
## ----------------------------------------------------------
\end{verbatim}

\backmatter

\hypertarget{references}{%
\chapter*{References}\label{references}}
\addcontentsline{toc}{chapter}{References}

\noindent

\setlength{\parindent}{-0.20in}
\setlength{\leftskip}{0.20in}
\setlength{\parskip}{8pt}

\hypertarget{refs}{}
\begin{CSLReferences}{1}{0}
\leavevmode\vadjust pre{\hypertarget{ref-angel2000}{}}%
Angel, E. (2000). \emph{Interactive computer graphics : A top-down approach with OpenGL}. Boston, MA: Addison Wesley Longman.

\leavevmode\vadjust pre{\hypertarget{ref-angel2001}{}}%
Angel, E. (2001a). \emph{Batch-file computer graphics : A bottom-up approach with QuickTime}. Boston, MA: Wesley Addison Longman.

\leavevmode\vadjust pre{\hypertarget{ref-angel2002a}{}}%
Angel, E. (2001b). \emph{Test second book by angel}. Boston, MA: Wesley Addison Longman.

\leavevmode\vadjust pre{\hypertarget{ref-Molina1994}{}}%
Molina, S. T., \& Borkovec, T. D. (1994). The {P}enn {S}tate worry questionnaire: Psychometric properties and associated characteristics. In G. C. L. Davey \& F. Tallis (Eds.), \emph{Worrying: Perspectives on theory, assessment and treatment} (pp. 265--283). New York: Wiley.

\leavevmode\vadjust pre{\hypertarget{ref-reedweb2007}{}}%
Reed~College. (2007). LaTeX your document. Retrieved from \url{http://web.reed.edu/cis/help/LaTeX/index.html}

\end{CSLReferences}


%% backmatter is needed at the end of the main body of your thesis to
%% set up page numbering correctly for the remainder of the thesis
\backmatter

%% Start the correct formatting for the appendices
% \appendix
%% Input each appendix here
% \input{./appendix_a}

%% Bibliography goes here (You better have one)
%% BibTeX is your friend

% \bibliographystyle{alpha}  % or use  abbrv to abbreviate first names and use numerical indices
\bibliographystyle{abbrv}  % or use  abbrv to abbreviate first names and use numerical indices
%% Add your BibTex file here (don't include the .bib)
\bibliography{./references}



%% Index go here (if you have one)
\end{document}
